\section{Énoncé}

Le but de ce projet est de simuler (de manière simplifiée) le chauffage d'un
objet (un aliment) par un four à micro-ondes en trois dimensions avec la
méthode des éléments finis.

Ce problème se résout en deux étapes. Tout d'abord, on résout l'équation de Helmoltz
afin de déterminer la magnitude du champ électromagnétique dans l'enceinte du four.
Ensuite, avec les données de cette magnitude, on résout l'équation de Poisson à
coefficients variables afin de déterminer la température de l'objet, et la température
de l'air qui entoure l'objet dans le four à micro-ondes.

% =============================================================================

\subsection{Le problème de Helmholtz}

Soit $\Omega \subset \mathbb{R}^3$ le domaine à l'intérieur du four.
Le problème de Helmholtz s'écrit alors comme suit :

Trouver $u$ une fonction définit sur l'ouvert $\Omega$ à valeurs dans $\mathbb{C}$
tel que
\[\begin{dcases}
    \omega^2 \mu u + \nabla \cdot \frac{1}{\varepsilon} \nabla u = 0 \qquad
    & \text{dans } \Omega \\
    u = g, \ \frac{\partial u}{\partial n} = 0 & \text{sur } \partial \Omega
\end{dcases}\]

%$\epsilon = 1$ dans l'air et $\epsilon = 4$ dans l'objet à cuire, et $g$ est
%une fonction qui modélise l'antenne du four. On suppose aussi que $\omega = 1$ et que
%$\mu \in \mathbb{C}$.

% TODO - mu et epsilon sont plutot des fonctions - ou ca ce fait plus tard........

où, pour $x \in \Omega$ et $\tilde{x} \in \partial\Omega$,

\begin{itemize}

    \item $u(x)$ est une fonction à valeurs complexes qui représente l'amplitude
    du champ électromagnétique.

    \item $\omega > 0$ est la vitesse angulaire des ondes électromagnétique émis
    par l'antenne (magnétron) du four.

    \item $\mu > 0$ est la perméabilité magnétique relative, qui
    caractérise la résistance d'un milieu à la création d'un champ magnétique.

    \item $\epsilon(x)$ est la (éventuellement complexe) permittivité relative,
    qui caractérise la polarisabilité d'un milieu sous l'effet d'un champ électrique.

    \item $g(\tilde{x})$ est une fonction qui représente l'antenne (magnétron), qui est
    la source du champ électromagnétique. Dans un four à micro-ondes l'antenne
    se trouve à l'extériur de l'enceinte, et donc on peut modéliser l'antenne
    comme une condition aus limites restreint à la paroi de l'enceinte.

%    \item $g : \partial \Omega \rightarrow \mathbb{R}$ est une fonction qui
%    représente la

\end{itemize}

% =============================================================================

\subsection{Le problème de Poisson à coefficients variables}

L'équation de Poisson s'écrit comme suit :

Trouver $v$ une fonction définit sur l'ouvert $\Omega$ à valeurs dans $\mathbb{R}$
tel que
\[\begin{dcases}
      -\nabla \cdot K \nabla v = f \qquad & \text{dans } \Omega \\
      v = 0 & \text{sur } \partial \Omega
\end{dcases}\]

où, pour $x \in \Omega$,

\begin{itemize}

    \item $v(x)$ est une fonction à valeurs réelles qui représente la température
    de l'air à l'intérieur de l'enceinte et la température de l'objet à cuire.

    \item $K(x) > 0$ est la conductivité thermique, qui caractérise le degré de
    transfert de chaleur dans un milieu.

    \item $f(x)$ est la source de chaleur. Pour modéliser le four
    à micro-onde on pose que la puissance de cette source est égale à la
    magnitude (au carré) du champ magnétique : $f = u \overline{u}$,
    où $u$ est la solution de l'équation de Helmholtz.

\end{itemize}
