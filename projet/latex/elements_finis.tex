\section{Éléments finis (complexes)}

Afin de résoudre numériquement les problèmes de Helmholtz et Poisson
on se sert de la méthode des éléments finis et la méthode de Galerkine.
On emploie l'espace des fonctions $\mathbb{P}_1$-Lagrange comme
fonctions de forme, et, les problèmes étant en 3D, des tétraèdres comme
éléments. On cherche alors une solution approchée dans l'espace donné
par
\begin{align}
    V_h = \{v \in C^0(\Omega), v|_\tau \in \mathbb{P}_1(\tau)
            \forall \tau \in T_h \},
\end{align}
%
où $T_h$ est l'ensemble de tétraèdres du maillage.

Soit maintenant $\mathcal{S}_h$ l'ensemble des sommets du maillage et
$\{\phi_s, s \in \mathcal{S}_h\}$ une base de $V_h$ telle que
\[\phi_s(s') =
    \begin{dcases}
        1 \text{ si } s = s' \\
        0 \text{ sinon}
    \end{dcases}
\]

Étant à valeurs complexes, la solution de l'equation de Helmholtz s'écrit
alors
\begin{align}
    u_h(x) = \sum_{s \in \mathcal{S}_h} u_s \phi_s(x) \qquad u_s \in \mathbb{C}
\end{align}

Par contre, la solution de l'équation de Poisson est à valeurs réelles.
Sa solution s'écrit donc
\begin{align}
    v_h(x) = \sum_{s \in \mathcal{S}_h} v_s \phi_s(x) \qquad v_s \in \mathbb{R}
\end{align}

\textbf{Remarque} \quad La solution de l'équation de Helmholtz est à valeurs
complexes mais on n'a pas besoin des fonctions de forme complexes
(cf.~\cite{hecht_elements_finis}). Pourtant, il faut introduire quelque part
des paramètres complexes pour faire en sorte que la solution soit complexe.
On va alors convertir la permittivité relative $\epsilon$ en valeur complexe.
