\section{Formulations Variationnelles}

Avant de résoudre numériquement ces deux problèmes avec la méthode des
éléments finis il faut tout d'abord montrer l'existence et l'unicité des solutions.

Pour l'equation de Poisson cette tache se fait plutôt facilement grace au théorème
de Lax-Milgram. L'equation de Helmholtz, par contre, nécessite un peu plus
d'attention.

% =============================================================================

\subsection{Équation de Poisson}

Soit $\Omega$ un ouvert borné, $f \in L^2(\Omega)$, $v \in C_0^\infty(\Omega)$,
et $w \in C_0^\infty(\Omega)$. On multiplie les 2 côtés de l'équation de Poisson
par $w$ et on les intègre. Cela nous donne
\[
    -\into w \nabla \cdot K \nabla v = \into fw
\]

Comme $v \in C_0^\infty(\Omega)$ et $w \in C_0^\infty(\Omega)$ on peut donc se
servir d'une formule de Green, ce qui nous amène à
\[
    \into K \nabla v \cdot \nabla w
    - \intob w K \nabla v \cdot n = \into fw
\]

Par définition $w$ est nulle sur $\partial \Omega$. D'où on en tire
\[
    \into K \nabla v \cdot \nabla w = \into fw
\]

Comme $C_0^\infty(\Omega)$ est dense dans $H_0^1(\Omega)$, on suppose maintenant
que $u \in H_0^1(\Omega)$. Par passage à la limite on en tire la formulation
variationnelle suivante :
\[\begin{dcases}
    \text{Trouver } v \in H_0^1(\Omega) \text{ telle que} \\
    a(v, w) = \ell(w) \forall w \in H_0^1(\Omega),
\end{dcases}\]
%
où
\begin{align}
    a(v, w) &= \into K \nabla v \cdot \nabla w \\
    \ell(w) &= \into fw
\end{align}

La continuité de la forme bilinéaire $a(v, w)$ et de la forme linéaire
$\ell(w)$ est évidente.

Pour démontrer la coercivité de $a(v, w)$ on note tout d'abord que, par l'inégalite
de Poincaré, il existe une constante $C > 0$ telle que
\begin{align}
    \norme{v}^2_{H^1(\Omega)} \le C \norme{\nabla v}^2_{L^2(\Omega)}
    \quad \forall v \in H_0^1(\Omega)
\end{align}

En notant $K_{\min} > 0$ le minimum de $K$, il s'ensuit que
\begin{align}
    a(v, v) = \into K \nabla v \cdot \nabla v
    \ge K_{min} \norme{\nabla v}^2_{L^2(\Omega)}
    \ge \frac{K_{min}}{C} \norme{v}^2_{H^1(\Omega)}
\end{align}

On a ainsi montré la coercivité de $a(v, w)$. Donc, d'après le théorème
de Lax-Milgram, il existe une unique solution du problème variationnelle.

% =============================================================================

\subsection{Équation de Helmholtz}

Afin de montrer l'existence et l'unicité de la solution de l'équation de
Helmholtz, il faut tout d'abord l'équation reformuler en problème de
conditions aux limites homogènes.

Pour réaliser cet objectif on étend la fonction $g$ de sorte qu'elle soit
définie sur tout $\overline \Omega$. On introduit donc la function
$\tilde{g} \in H^2(\overline \Omega)$, où $\tilde{g}(x) = g(x)$
pour tout $x \in \partial \Omega$.

Soit maintenant $\ut = u - \tilde{g}$. On a donc
$\ut = 0$ sur $\partial \Omega$, et notre problème s'écrit
\[\begin{dcases}
      \omega^2 \mu \ut + \nabla \cdot \frac{1}{\varepsilon} \nabla \ut =
      -\omega^2 \mu \tilde{g} - \nabla \cdot \frac{1}{\varepsilon} \nabla \tilde{g} \qquad
      & \text{dans } \Omega \\
      \ut = 0 & \text{sur } \partial \Omega
\end{dcases}\]

Avec $f = -\omega^2 \mu \tilde{g} - \nabla \cdot \frac{1}{\varepsilon} \nabla \tilde{g}$,
on en tire un problème de conditions aux limites de Dirichlet homogène :
\[\begin{dcases}
      \omega^2 \mu \ut + \nabla \cdot \frac{1}{\varepsilon} \nabla \ut = f
      \qquad & \text{dans } \Omega \\
      \ut = 0 & \text{sur } \partial \Omega
\end{dcases}\]

En suivant les mêmes étapes que pour l'equations de Poisson,
%et on notant que $\frac{\partial \ut}{\partial n} = 0$ sur le bord du domaine,
on en tire la formulation variationnelle pour l'equation de Helmholtz :
\[\begin{dcases}
    \text{Trouver } \ut \in H_0^1(\Omega) \text{ telle que} \\
    a(\ut, w) = \ell(w) \forall w \in H_0^1(\Omega),
\end{dcases}\]
%
où
\begin{align}
    a(\ut, w) &= \omega^2 \mu \into \ut w
    -\into \frac{1}{\varepsilon} \nabla \ut \cdot \nabla w \\
    \ell(w) &= \into fw
\end{align}

La continuité de la forme bilinéaire $a(v, w)$ et de la forme linéaire
$\ell(w)$ est, comme avant, évidente. Par contre, la coercivité de
$a(\ut, w)$ s'avére plus compliqué.

En principe, avec une estimation précise de la constante de Poincaré,
on pourrait imposer des contraintes aux constantes de sorte que $a(\ut, w)$ soit
coercive. Mais de telles contraintes nous laisserait probablement avec un four
à micro-ondes qui ne chauffe rien du tout !

% ==========================================================================

Au lieu de s'attaquer à ce problème d'analyse on va raisonner "comme des
ingénieurs". C'est-á-dire, si la solution numérique se comporte bien, alors
tout va bien !

Pour un traitement sérieux des questions d'existence et d'unicité
de l'équation de Helmholtz, et en particulier l'alternative de Fredholm,
voir les références \cite{fredholm1}, \cite{fredholm2}, \cite{fredholm3},
\cite{fredholm4}, \cite{fredholm5}, \cite{fredholm6}, et \cite{fredholm7}.


